  

%

% Preambel

%

\documentclass[a4paper]{article}

%

\title{vars: S3 classes and methods for estimating VAR and SVAR models}

\author{Dr. Bernhard Pfaff, Kronberg im Taunus}

%

% Bibliographystle

\usepackage{harvard}

\citationstyle{agsm}

\bibliographystyle{agsm}

\harvardyearparenthesis{square}

\usepackage{amsmath}

%

% Begin of document

%

\usepackage{/usr/local/lib/R/share/texmf/Sweave}
\begin{document}

%

% Instructions for vignette

% \VignetteIndexEntry{vars: A S3 class and methods for estimating VAR and SVAR models}

% \VignetteDepends{MASS, strucchange}

%\VignetteKeywords{VAR, SVAR, lag selection, diagnsotics, forecasting, causality , FEVD and IRA.}

%\VignettePackage{vars}





\maketitle

%

\abstract{The utilisation of the functions contained in the package `vars' are explained by employing a data set of the Canadian economy. The package's scope includes functions for estimating vector autoregressive (henceforth: VAR) and structural vector autoregressive models (henceforth: SVAR). In addition to the two cornerstone functions \texttt{VAR()} and \texttt{SVAR()} for estimating such models, functions for diagnostic testing, estimation of restricted VARs, prediction of VARs, causality analysis, impulse response analysis (henceforth: IRA) and forecast error variance decomposition (henceforth: FEVD) are provided too. In each section, the underlying concept and/or method is briefly explained, thereby drawing on the exibitions in \citeasnoun{LUE2006}, \citeasnoun{LUE2004}, \citeasnoun{LUE1997}, \citeasnoun{HAM1994} and \citeasnoun{WAT1994}.\\

\par

The package's code is purely written in \texttt{R} and S3-classes with methods have been utilised. It is shipped with a NAMESPACE and a ChangeLog file. It has dependencies to \texttt{MASS} (see \citeasnoun{MASS}) and \texttt{strucchange} (see \citeasnoun{strucchange}).} 

%

\tableofcontents

% 1st Sweave

